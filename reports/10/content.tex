\section{Цель работы}
Цель: изучение и приобретение навыков работы с программными межсетевыми экранами. 

Задачи:
\begin{itemize}
    \item Изучение возможностей персональных межсетевых экранов;
    \item Конфигурирование персонального межсетевого экрана;
    \item Управление соединениями с помощью персонального межсетевого экрана.
\end{itemize}


\section{Ход работы}
\subsection{Private Firewall}
Загрузим программу по ссылке https://www.comss.ru/download/page.php?\linebreak id=790
Установим Private Firewall на виртуальную машину. Окно программы приведено на рисунке 1.

\image{11.png}{Окно программы}{1}
\FloatBarrier

\subsection{Настройки межсетевого экрана}
В верхнем углу есть основной переключатель режимов работы межсетевого экрана. При нажатии на зелёную кнопку весь трафик разрешается, при нажатии на красную — блокируется. В жёлтом режиме трафик фильтруется по правилам, которые можно настроить ниже.

В окне программы есть пять подменю для настройки правил.

\subsubsection{Main menu}
\begin{itemize}
    \item Доступ в интернет — уровень защиты при работе в интернете. Высокий уровень (рекомендуемый) — подходит для небезопасной среды, отключает все протоколы, использование которых не предусмотрено в интернете. Низкий уровень — для использования интернета в безопасной среде. Также можно самостоятельно задать разрешения (и посмотреть, что отключено в High) в пользовательском режиме.
    \item Безопасность в сети — уровень защиты при работе в интернете и локальной сети. Нужен для настройки работы сетевых дисков и принтеров. Низкий уровень (рекомендуемый) — устройства в локальной сети доступны (адрес локальной сети можно задать дополнительно). Высокий уровень — многие протоколы отключены в локальной сети, может привести к проблемам в сетевом взаимодействии.
    \item Доверенные IP/сайты — сайты и адреса, с которых разрешён доступ к компьютеру. Можно задать адрес устройства, подсеть или URL.
    \item Заблокированные IP/сайты — сайты и адреса, с которых запрещён доступ к компьютеру.
\end{itemize}

\subsubsection{Applications}
В разделе перечислен список приложений, которые пытались подключиться к интернету. Для каждого приложения существует список правил, которые можно настроить через контекстное меню. Также можно временно включить или отключить фильтрацию трафика для приложения (Рисунок 2).

\image{12.png}{Applications}{1}
\FloatBarrier

При добавлении правила можно указать очерёдность применения правил, указывать тип правила, протокол, локальные и удалённые порты, а также в какой режим работы добавить данное правило (Рисунок 3).

\image{121.png}{Applications}{1}
\FloatBarrier

\subsubsection{Process monitor}
Раздел позволяет мониторить процессы, запущенные в системе, и отслеживать системные вызовы. Для каждого процесса можно настроить правила, по которым одни системные вызовы буду доступны, а другие нет. Можно выбрать уровень защиты: в среднем режиме будут отслеживаться только процессы, относящиеся к приложениям, перечисленным в предыдущем меню программы, в высоком режиме будут отслеживаться все процессы в системе (Рисунок 4). Доступные системные вызовы перечислены на рисунке 5.

\image{13.png}{Process monitor}{1}
\image{131.png}{Process monitor}{1}
\FloatBarrier

\subsubsection{Firewall Log}
Раздел отображает входящие и исходящие пакеты, которые были заблокированы межсетевым экраном. В списке отображаются время и дата, локальный IP, удалённый IP, протокол и приложение, к которому относится пакет. Также указано, в каком направлении следовал пакет (Рисунок 6).

\image{14.png}{Firewall Log}{1}
\FloatBarrier

\subsubsection{Port tracking}
Раздел отслеживает все порты в системе и защищает их от незапланированного проникновения. Кроме того, фаервол скрывает порты, делая их невидимыми для внешних устройств (Stealth). В таблице отображаются имя приложения, PID, протокол и адреса (Рисунок 7).

\image{15.png}{Port tracking}{1}
\FloatBarrier
\clearpage


\subsection{Создание сетевых соединений и правил}

Протестируем различные программы, использующие доступ к сети.

\subsubsection{Git}

Попробуем воспользоваться утилитой git, чтобы скачать репозиторий с гитхаба. Запустим git bash и пронаблюдаем в логах фаервола, какие пакеты заблокированы. На рисунке 8 видно, что это программа git-remote-https.exe. 

\image{21.png}{Трафик запрещён}{1}

Создадим правило для разрешения трафика. Правило применим к нашему приложению (Рисунок 9). 

\image{22.png}{Создание правила на приложение}{1}

После применения правила доступ к гитхабу появился и мы смогли загрузить репозиторий (Рисунок 10).

\image{23.png}{Трафик разрешён}{1}
\FloatBarrier
\clearpage


\subsubsection{Windows Media Player}

Попробуем воспользоваться приложением Windows Media Player. В нём встроен доступ к интернет-магазинам музыки. Открыв приложение, видим запрос на исходящий трафик, который перехватил фаервол (Рисунок 11). 

\image{31.png}{Запрос на исходящий трафик}{1}

Создадим правило для разрешения трафика. Правило применим к нашему приложению. Настроим разрешение на порт 80, на котором расположен интернет-магазин (Рисунок 12). Видим, что в окне программы появилось сообщение, загруженное с сервера microsoft.com. 

\image{32.png}{Правило на порт}{1}

Теперь переведём правило в блокирующий режим (контекстное меню > блокировать). После применения правила доступ к интернет-магазину пропал (Рисунок 13).

\image{33.png}{Трафик заблокирован}{1}
\FloatBarrier
\clearpage


\subsubsection{ssh}

Воспользуемся утилитой ssh для удалённого доступа к инстансу docker-\linebreak playground. Для этого из git bash выполним команду ssh с адресом сервера (Рисунок 14).
Видим, что в окне фаервола появилась строка ssh.exe (имя утилиты, которую мы запустили), а также что доступ к хосту у нас имеется (мы смогли сделать попытку установить сессию и сервер не принял наш ключ).

\image{41.png}{Трафик разрешён}{1}

Переключим правило в режим Deny. Запустим ssh с флагом -v для просмотра расширенных логов. Видим, что соединение с сервером не может быть установлено (Рисунок 15).

\image{42.png}{Трафик запрещён}{1}

Откроем правило и настроим фильтрацию, разрешающую доступ к порту 22, который является стандартным для протокола (Рисунок 16). После этого доступ к серверу вновь появился.

\image{43.png}{Трафик разрешён}{1}

Снова заблокируем доступ на порт 22. Результат виден на рисунке 17.

\image{44.png}{Трафик заблокирован}{1}
\FloatBarrier
\clearpage


\subsection{Сканирование системы}

В качестве дополнительного задания выполним сканирование виртуальной машины, защищённой фаерволом, с помощью утилиты nmap. 

При первом запуске сканирования (nmap 192.168.56.101) видим, что сканер не смог выполнить пинг до хоста. Делаем вывод, что фаервол блокирует входящие icmp-запросы.  После запуска сканирования с ключом -Pn видим, как фаервол отобразил запрос на входящее соединение (Рисунок 18).

\image{52.png}{Запросы}{1}

После запрета на эти запросы повторяем сканирование. В результате сканер видит 4 открытых порта и 1 закрытый.

\image{53.png}{Заблокированные сервисы}{1}

Переведём фаервол в зелёный режим — теперь весь трафик разрешён. Видим, что появились несколько новых открытых портов (Рисунок 20).

\image{54.png}{Трафик разрешён}{1}

Команда ping также срабатывает в таком режиме (Рисунок 21).

\image{55.png}{Команда ping работает}{1}

Вернём фаервол в режим фильтрации. Команда ping больше не работает (Рисунок 22).

\image{56.png}{Команда ping не работает}{0.7}
\FloatBarrier
\clearpage


\section{Выводы о проделанной работе}
Я изучил и приобрёл навыки работы с программными межсетевыми экранами, различные конфигурации персонального межсетевого экрана, а также управление соединениями с помощью персонального межсетевого экрана.

\clearpage