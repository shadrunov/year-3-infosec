\section{Цель работы}
Цель: освоение основ анализа защищённости сетевого компьютера с
использованием пассивных и активных методов.

\section{Ход работы}
\subsection{Создание ВМ}
Для выполнения работы используются две виртуальные машины, на одну из которых устанавливается Сканер-ВС, а на другую — тестовая уязвимая операционная система Metasploitable. (Рисунок 1).

\image{11.png}{Создание ВМ}{1}
\FloatBarrier

Проверим настройки сети. У метасплойта адрес 192.168.56.102, у сканера — .103 (Рисунок 2).

\image{12.png}{Адреса ВМ}{1}
\FloatBarrier
\clearpage


\subsection{Лабораторная работа 1. Сканирование портов}
В этой работе изучим процесс сканирования портов. Для этого создадим новый проект и добавим задачу (Рисунок 3). В качестве цели зададим подсеть.

\image{14.png}{Новая задача}{1}
\image{15.png}{Новая задача}{1}
\FloatBarrier

Запустим задачу (Рисунок 5). В результате на вкладке Порты увидим открытые порты метасплойта (Рисунок 6). На вкладке Хосты отображаются найденные устройства (Рисунок 7).

\image{16.png}{Выполнение задачи}{1}
\image{17.png}{Открытые порты}{1}
\image{18.png}{Хосты в сети}{1}
\FloatBarrier

Добавим новую задачу и в расширенных настройках выберем определять версию сервисов (Рисунок 8). В результате на вкладке Порты увидим версии сервисов, работающих на этих портах, например, httpd 2.2.8 на порту 80 (Рисунок 9).

\image{19.png}{Определять версию сервисов}{1}
\image{110.png}{Версии сервисов}{1}
\FloatBarrier

Также добавим задачу на определение топологии сети. Для этого в расширенных настройках выберем трассировка пути (Рисунок 10). В результате на вкладке Топология появится топология нашей сети (Рисунок 11). В центре находится сканер, к нему подключен узел 192.168.56.102 (метасплойт), ещё устройства помечены пунктирной линией. Это DHCP-сервер гипервизора VirtualBox и хостовой адаптер.

\image{111.png}{Трассировка пути}{1}
\image{112.png}{Топология сети}{1}
\FloatBarrier
\clearpage



\subsection{Лабораторная работа 2. Поиск уязвимостей}
В этой работе изучим принцип работы сканеров безопасности. Для этого перейдём на панель Поиск уязвимостей (Рисунок 12). Создадим новую задачу и выберем цель из активов, оставшихся от предыдущего сканирования (Рисунок 13).

\image{21.png}{Поиск уязвимостей}{1}
\image{22.png}{Выбор целей из активов}{1}
\FloatBarrier

Запустим задачу и подождём некоторое время, пока сканирование завершится. После этого на вкладке Уязвимости появится список найденых на сервере уязвимостей с указанием их критичности (Рисунок 14).

\image{23.png}{Уязвимости}{1}
\FloatBarrier
\clearpage



\subsection{Лабораторная работа 3. Сетевой аудит паролей}
В этой работе проведём сетевой аудит паролей Metasploitable 2. Для этого перейдём на панель Эксплуатация (Рисунок 15). Создадим новую задачу и выберем цель из активов, оставшихся от предыдущего сканирования. Выберем сервис, который будет сканироваться, — ftp (Рисунок 16).

\image{31.png}{Эксплуатация}{1}
\image{32.png}{Выбор сервиса ftp}{1}
\FloatBarrier

Добавим юзернеймы и пароли в соответствующих вкладках (Рисунки 17-18). 

\image{33.png}{Добавляем пользователей}{1}
\image{34.png}{Добавляем пароли}{1}
\FloatBarrier

По завершению задачи (Рисунок 19) перейдём на вкладку Сетевой аудит паролей (Рисунок 20). Видим, что система подобрала пароли для сервиса ftp, этот сервис является уязвимостью для метасплойта.

\image{35.png}{Задача завершена}{1}
\image{36.png}{Сетевой аудит паролей}{1}
\FloatBarrier

Аналогично просканируем сервис smb (Рисунок 21). Сканирование отработало, однако новые пароли не отобразились на вкладке Сетевой аудит паролей. Это значит, что подобрать пароли к сервису smb не получилось.

\image{37.png}{Задача завершена}{1}
\image{38.png}{Сетевой аудит паролей - smb}{1}
\FloatBarrier
\clearpage



\subsection{Лабораторная работа 4. Поиск подходящих эксплойтов}
В этой работе научимся использовать веб-интерфейс Сканер-ВС для поиска
эксплойтов. Для этого перейдём на панель Эксплуатация (Рисунок 23). Создадим новое сканирование на вкладке Поиск эксплойтов (Рисунок 24). Выберем тип сканирования (Рисунок 25). Результат отобразится на вкладке Поиск эксплойтов (Рисунок 26)

\image{41.png}{Эксплуатация}{1}
\image{42.png}{Поиск эксплойтов}{1}
\image{43.png}{Выбираем цель}{1}
\image{44.png}{Результаты}{1}
\FloatBarrier

Создадим отчёт. Для этого откроем вкладку Отчёты и создадим задачу. Выберем полный отчёт и скачаем его в формате PDF (Рисунки 27-29). Некоторые страницы отчёта приведены в приложении А.

\image{46.png}{Создаём отчёт}{1}
\image{47.png}{Отчёт можно скачать в разных форматах}{1}
\image{48.png}{Скачиваем отчёт}{1}
\FloatBarrier

\clearpage


\section{Выводы о проделанной работе}
Я освоил основы работу со сканером уязвимостей Сканер-ВС, изучил уязвимости в системе метасплойт с использованием активных и пассивных методов.

\clearpage