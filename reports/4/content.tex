\section{Цель работы}
Цель: работа с программно-аппаратными средствами криптографической защиты информации.

\section{Ход работы}
\subsection{GPG}

В Linux также есть средства для работы с ключами PGP. На их основе, например, строятся доверительные отношения при распространении ПО: разработчик подписывает дистрибутивы своим ключом, и клиенты могут проверить подлинность файла с помощью опубликованного второго ключа.

В Arch уже установлен пакет gnupg (\url{https://wiki.archlinux.org/title/GnuPG})для работы с ключами (Рисунок 1).

\image{11.png}{Версия gpg}{1}
\FloatBarrier

\subsubsection{Создание ключевой пары}
Создадим ключевую пару (Рисунок 2). Во время создания утилита просит выбрать тип шифрования, длину ключа, срок действия, имя, адрес и комментарий. После этого появляется приглашение ввести пароль для ключа (Рисунок 3). После этого утилита сообщает, что публичный и секретный ключи созданы и подписаны (Рисунок 4).

\image{12.png}{Версия gpg}{1}
\image{13.png}{Версия gpg}{1}
\image{14.png}{Версия gpg}{1}
\FloatBarrier

Можем увидеть созданные ключи в ~/.gnupg (Рисунок 5). 

pubring.kbx — контейнер с ключами, содержит один или несколько публичных ключей и сертификатов с метаданными. Этот файл позволяет импортировать публичный ключ в другие программы. 

Закрытые ключи находятся в папке private-keys-v1.d. В папке openpgp-revocs.d/ находится сертификат для отзыва ключей.

\image{15.png}{~/.gnupg}{1}
\FloatBarrier

Также ключи можно посмотреть в приложении Passwords and Keys (пакет \linebreak seahorse). 

\image{16.png}{Passwords and Keys}{1}
\image{17.png}{Passwords and Keys}{0.9}
\FloatBarrier
\clearpage


\subsubsection{Просмотр импортированных ключей}
Для этого выполним команду \texttt{gpg --list-keys} (Рисунок 8). Видно, что я импортировал ключ для установки Spotify, для драйвера eToken 5110 от Thales и для чего-то ещё. Также можно посмотреть имеющиеся у меня закрытые ключи командой \texttt{gpg --list-secret-keys} (Рисунок 9).

\image{21.png}{Public keys}{1}
\image{22.png}{Private keys}{1}
\FloatBarrier
\clearpage


\subsubsection{Экспорт ключей}
Для экспорта выполним команду \texttt{gpg --export --armor --output my-key.asc asshadrunov@gmail.com}. Файл появляется в директории (Рисунок 10). Внутри у него записаны байты в стандартной для ключей форме (Рисунок 11). 

\image{32.png}{Export key}{1}
\image{31.png}{Inside asc file}{1}
\FloatBarrier

Также утилита позволяет экспортировать ключ на публичные серверы ключей. Такой, например, предоставляет Ubuntu.
\clearpage


\subsubsection{Шифрование файла}
Для шифрования файла нужно импортировать публичный ключ получателя. Так как мой ключ уже импортирован, зашифруем файл file.txt для меня. Для этого выполняю команду \texttt{gpg --recipient asshadrunov@gmail.com --encrypt file.txt} (Рисунок 12). Зашифрованный файл появляется в директории (file.txt.gpg). 

\image{41.png}{Encrypted file}{1}
\FloatBarrier

Можем зашифровать файл в ASCII-виде с помощью опции \texttt{--armor} (Рисунок 13). Тогда его зашифрованное содержимое легко посмотреть и передать сообщением (Рисунок 14). 

\image{42.png}{ASCII file}{1}
\image{43.png}{ASCII file}{1}
\FloatBarrier
\clearpage


\subsubsection{Расшифрование файла}
Расшифрование произодится командой \texttt{ggpg --decrypt file.txt.asc} (Рисунок 12). Приглашение попросит ввести парольную фразу от ключа.

\image{51.png}{Decrypted file}{1}
\FloatBarrier
\clearpage

\subsection{TrueCrypt}
\subsubsection{Установка}

Установим TrueCrypt на виртуальную машину (Рисунок 16).

\image{61.png}{Installation}{1}
\FloatBarrier

\subsubsection{Том из файла}

Создадим том TrueCrypt. Для этого откроем окно программы и выберем Create Volume (Рисунок 17).

\image{62.png}{TrueCrypt}{0.85}
\FloatBarrier

На следующем шаге выбираем файловый контейнер, выбираем расположение, размер тома, задаём пароль (Рисунки 18-22).

\image{63.png}{Create container}{0.85}
\FloatBarrier

\image{64.png}{Location}{1}
\FloatBarrier

\image{65.png}{Volume size}{1}
\FloatBarrier

\image{66.png}{Password}{1}
\FloatBarrier

Теперь мы можем нажать Select File, а затем Mount, после чего том смонтируется к диску Q: (Рисунок 22). При этом этот диск отображается в системе (в Проводнике).

\image{67.png}{Mounted volume}{1}
\FloatBarrier

Если нажать Dismount, всё вернётся как было (Рисунок 23).

\image{68.png}{Dismount}{1}
\FloatBarrier

\clearpage


\subsubsection{Шифрование системного раздела}

Запустим System > Encrypt system partition. Далее выбираем область для шифрования (только раздел с Windows), опции шифрования, пароль. После этого мастер сгенерирует ключи и попросит записать их на диск. Это можно пропустить, если запустить мастера с ключом /n. После этого система предложит перезагрузить компьютер (Рисунки 24-28).

\image{71.png}{Encrypt system partition}{0.85}
\FloatBarrier

\image{72.png}{Encryption options}{1}
\FloatBarrier

\image{73.png}{Password}{1}
\FloatBarrier

\image{74.png}{Keys generated}{1}
\FloatBarrier

\image{75.png}{Prompt ro reboot}{1}
\FloatBarrier

При перезагрузке вместо загрузчика Windows мы увидим TrueCrypt Boot \linebreak Loader, который просит ввести пароль, чтобы расшифровать раздел с Windows (Рисунок 29).

\image{76.png}{TrueCrypt Boot Loader}{1}
\FloatBarrier

После успешного ввода пароля загрузка продолжается (Рисунок 30), а мастер сообщает нам, что проверка пройдена (Рисунок 31).

\image{77.png}{Windows loading}{1}
\FloatBarrier

\image{78.png}{Pretest completed}{1}
\FloatBarrier

После этого начинается шифрование диска. Прогресс виден на рисунке 32.

\image{79.png}{Disk encryption}{1}
\FloatBarrier

После этого раздел C: отображается в TrueCrypt как зашифрованный том (Рисунок 33).

\image{80.png}{System partition in TrueCrypt}{1}
\FloatBarrier

\clearpage


\section{Выводы о проделанной работе}
Я изучил работу с программно-аппаратными средствами криптографической защиты информации на примере программы для шифрования gpg и программы для шифрования томов TrueCrypt.

\clearpage