\section{Цель работы}
Цель: изучить принципы формирования паролей в операционных системах Windows начиная с Windows Server 2012 и Windows 8.1. Также изучить процесс создания надежных паролей, парольных фраз и политики паролей.

\section{Ход работы}
\subsection{Дамп паролей}
Мы будем взламывать пароли, хранящиеся в виде LM и NT хэшей. Для этого на виртуальной машине с Windows XP создадим нескольких пользователей (alex — администратор, user — ограниченный пользователь). После этого скачаем утилиту pwdump6 (Рисунок 1).

\image{11.png}{Создание пользователей}{1}
\FloatBarrier

По умолчанию в системе пароли хранятся в виде обоих хэшей. С помощью команды pwdump.exe localhost сделаем дамп паролей всех пользователей (Рисунок 2).

\image{12.png}{Дамп LM и NT хэшей}{1}
\FloatBarrier

Будем менять пароли по схеме: числовой (до 6 символов), числовой (6-10 символов), буквенный (до 6 символов), словарное слово (до 6 символов), буквенный (6-10 символов), словарное слово (6-10 символов), смешанный (буквы и цифры до 6 символов), смешанный (буквы, цифры, спецсимволы до 6 символов). Затем снова будем делать дамп. 

Также попробуем отключить хранение LM хэшей. Для этого отредактируем реестр, как показано на рисунках 3-4.

\image{13.png}{regedit}{1}
\image{14.png}{regedit}{1}
\FloatBarrier

После этого LM хэш больше не выводится в дампе, только NT (Рисунок 5).

\image{15.png}{Дамп только NT хэшей}{1}
\FloatBarrier

Результаты всех дампов приведены в таблице 1.

\begin{table}[]
    \centering
    \caption{NT и LM хэши}
    \begin{tabular}{ll}
    \textbf{LM}   & \textbf{}                        \\
    \textbf{alex} &                                  \\
    1234          & B757BF5C0D87772FAAD3B435B51404EE \\
    1234567       & 0182BD0BD4444BF8AAD3B435B51404EE \\
    abcd          & E165F0192EF85EBBAAD3B435B51404EE \\
    qwerty        & 598DDCE2660D3193AAD3B435B51404EE \\
    password      & E52CAC67419A9A224A3B108F3FA6CB6D \\
    qpoi3         & 875FBACE27462DB2AAD3B435B51404EE \\
    wer4          & D150630C44CEB678AAD3B435B51404EE \\
    \textbf{user} &                                  \\
    5678          & A7F94BDC54C75446AAD3B435B51404EE \\
    7654321       & 9EE0D521C88B2C76AAD3B435B51404EE \\
    efgh          & 6391AAA063992CA8AAD3B435B51404EE \\
    pass          & B267DF22CB945E3EAAD3B435B51404EE \\
    windows       & 4EFC971E2C6A11F0AAD3B435B51404EE \\
    dd4er         & D6734695717A0410AAD3B435B51404EE \\
    rew3@         & 28EF1584D1C91091AAD3B435B51404EE \\
                  &                                  \\
    \textbf{NT}   &                                  \\
    \textbf{alex} & \textbf{}                        \\
    1234          & 7CE21F17C0AEE7FB9CEBA532D0546AD6 \\
    1234567       & 328727B81CA05805A68EF26ACB252039 \\
    abcd          & EB4FF39B74B0CBCE20A4F62DBD1E3585 \\
    qwerty        & 2D20D252A479F485CDF5E171D93985BF \\
    password      & 8846F7EAEE8FB117AD06BDD830B7586C \\
    qpoi3         & E6707CAD017AA910A75D5B2386DB5B87 \\
    wer4          & E5AFF6A771E8445FFDFB512372C567FB \\
    \textbf{user} &                                  \\
    5678          & 4D3BD16C0B87D0FE19DB8AD598D778FF \\
    7654321       & 8BB80565A55DEAA6E1847DC1BC3505FA \\
    efgh          & A0B95A38B5E679942183EF613FCB3C18 \\
    pass          & 36AA83BDCAB3C9FDAF321CA42A31C3FC \\
    windows       & A2345375A47A92754E2505132ACA194B \\
    dd4er         & 0DE51E26F132968A78E39A36BF5D1C53 \\
    rew3@         & 4B57AB7B11E843943831D0A67CCE887C \\
                  &                                  \\
                  &                                  \\
                  &                                 
    \end{tabular}
    \end{table}
\clearpage

\subsection{Подбор паролей}
Теперь нужно подобрать пароли, которые отвечают полученным хэшам. Для этого воспользуемся утилитой hashcat. Аргументы, которые нам понадобятся:
\begin{itemize}
    \item -m 3000 — тип хэша. 3000 означает LM, 1000 — NTLM (NT);
    \item -a 3 — тип атаки. 3 означает атаку по маске, 0 — атака по словарю;
    \item -i — перебирать все возможные длины паролей;
    \item --outfile — выходной файл;
    \item --outfile-format=1,2,5,6 — формат выходного файла. 5 и 6 добавляют вывод метки времени, 1 и 2 — хэш и пароль;
    \item --potfile-path — путь к файлу кэша, куда hashcat помещает взломанные хэши;
    \item маска — мы будем использовать маску ?a?a?a?a?a?a?a для взлома LM, так как длина пароля в этом хэше не превышает 7 символов и мы знаем, что он состоит из букв, цифр и символов.
\end{itemize}

На рисунках 6-7 показан запуск hsahcat для взлома первого хэша.
\image{21.png}{hashcat}{1}
\image{22.png}{hashcat}{1}
\FloatBarrier

На рисунке 8 виден выходной файл. Видно, что перебор пароля занял меньше 1 секунды.

\image{23.png}{hashcat}{0.7}
\FloatBarrier

Для словарных паролей попробуем атаку по словарю (-a 0). Для этого воспользуемся словарём xato-net-10-million-passwords-10000.txt. Результат на рисунках ниже:
\image{24.png}{Скачиваем список паролей}{0.9}
\image{25.png}{Запускаем утилиту}{0.9}
\image{26.png}{Готово}{0.9}
\FloatBarrier
Повторим процедуру для остальных LM хэшей. Результат в таблице 2.

\begin{table}[]
    \centering
    \caption{LM хэши}
    \begin{tabular}{llll}
    \textbf{LM}   & \textbf{}                        &                &                  \\
    \textbf{alex} &                                  & атака маской   & атака по словарю \\
    1234          & B757BF5C0D87772FAAD3B435B51404EE & 0              & 0                \\
    1234567       & 0182BD0BD4444BF8AAD3B435B51404EE & 1944           & 0                \\
    abcd          & E165F0192EF85EBBAAD3B435B51404EE & 0              & 0                \\
    qwerty        & 598DDCE2660D3193AAD3B435B51404EE & 16             & 0                \\
    password      & E52CAC67419A9A224A3B108F3FA6CB6D & 1018           & 0                \\
    qpoi3         & 875FBACE27462DB2AAD3B435B51404EE & 5              & -                \\
    wer4!         & D150630C44CEB678AAD3B435B51404EE & 5              & -                \\
    \textbf{user} &                                  &                &                  \\
    5678          & A7F94BDC54C75446AAD3B435B51404EE & 0              & 0                \\
    7654321       & 9EE0D521C88B2C76AAD3B435B51404EE & 1116           & 0                \\
    efgh          & 6391AAA063992CA8AAD3B435B51404EE & 0              & -                \\
    pass          & B267DF22CB945E3EAAD3B435B51404EE & 0              & 0                \\
    windows       & 4EFC971E2C6A11F0AAD3B435B51404EE & $\sim$10 часов & 0                \\
    dd4er         & D6734695717A0410AAD3B435B51404EE & 0              & -                \\
    rew3@         & 28EF1584D1C91091AAD3B435B51404EE & 3              & -                \\
                  &                                  &                &                 
    \end{tabular}
\end{table}

Аналогично для NT хэшей используем код 1000 (Таблица 3):

\begin{table}[]
    \centering
    \caption{NT хэши}
    \label{tab:my-table}
    \begin{tabular}{llll}
    \textbf{NT}   &                                  &                &                  \\
    \textbf{alex} & \textbf{}                        & атака маской   & атака по словарю \\
    1234          & 7CE21F17C0AEE7FB9CEBA532D0546AD6 & 0              & 0                \\
    1234567       & 328727B81CA05805A68EF26ACB252039 & 1817           & 0                \\
    abcd          & EB4FF39B74B0CBCE20A4F62DBD1E3585 & 0              & 0                \\
    qwerty        & 2D20D252A479F485CDF5E171D93985BF & 100            & 0                \\
    password      & 8846F7EAEE8FB117AD06BDD830B7586C & 1244           & 0                \\
    qpoi3         & E6707CAD017AA910A75D5B2386DB5B87 & 5              & -                \\
    wer4!         & E5AFF6A771E8445FFDFB512372C567FB & 10             & -                \\
    \textbf{user} &                                  &                &                  \\
    5678          & 4D3BD16C0B87D0FE19DB8AD598D778FF & 0              & 0                \\
    7654321       & 8BB80565A55DEAA6E1847DC1BC3505FA & 1911           & 0                \\
    efgh          & A0B95A38B5E679942183EF613FCB3C18 & 0              & -                \\
    pass          & 36AA83BDCAB3C9FDAF321CA42A31C3FC & 0              & 0                \\
    windows       & A2345375A47A92754E2505132ACA194B & $\sim$10 часов & 0                \\
    dd4er         & 0DE51E26F132968A78E39A36BF5D1C53 & 5              & -                \\
    rew3@         & 4B57AB7B11E843943831D0A67CCE887C & 5              & -                \\
                  &                                  &                &                 
    \end{tabular}
\end{table}

\subsection{Анализ}
\subsubsection{Влияние метода шифрования}
В использованных нами методах атаки (по маске и по словарю) тип шифрования не оказывает существенного влияния, так как сложность взлома определяется количеством перебираемых вариантов, а время, затрачиваемое на хэширование, мало и примерно одинаково.

Атака по словарю в обоих случаях работает за малое время, если пароль присутствует в словаре, либо не работает вовсе.

\subsubsection{Влияние длины пароля}
Длина пароля в случае атаки по маске является определяющим фактором. Чем длиннее пароль, тем сложнее его подобрать. Пароли из 4 символов подбираются очень быстро, пароли из 5 символов — за единицы секунд, из 6 символов — за десятки секунд, из 7 символов — в пределах часа. Так происходит, потому что количество вариантов при добавлении одного символа возрастает в количество раз, равное мощности алфавита.

При подборе по словарю длина пароля не влияет на успех атаки.

\subsubsection{Наличие спецсимволов}
При переборе по маске наличие спецсимволов делает невозможным отгадывание пароля с простой маской, не учитывающей эти символы. Такая маска включена по умолчанию. Так как мы расширили маску до всех служебных символов на любых позициях, то в нашем случае они не оказывали влияния на успешность атаки.

При подборе по словарю спецсимволы являются хорошей защитой, так как в словаре чаще встречаются обычные слова. Но есть вариант гибридной атаки, когда маска прибавляется к словарным словам.


\clearpage

\section{Выводы о проделанной работе}
Я изучил принципы формирования паролей в операционных системах Windows начиная с Windows Server 2012 и Windows 8.1. Также изучил процесс создания надежных паролей, парольных фраз и политики паролей, а также освоил утилиту hashcat и pwdump6.

\clearpage