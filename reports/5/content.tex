\section{Цель работы}
Цель: Приобретение навыков работы с приложениями, применяющимися в сфере обеспечения информационной безопасности.

\section{Ход работы}

\subsection{Описание программы}
nmap (сокращение от network mapper) — утилита, предназначенная для разнообразного настраиваемого сканирования IP-сетей с любым количеством объектов, определения состояния объектов сканируемой сети (портов и соответствующих им служб). Изначально программа была реализована для систем UNIX, но сейчас доступны версии для множества операционных систем. (Википедия).

Программа поставляется в виде консольной утилиты на unix-системах и с графическим интерфейсом Zenmap на Windows. Приведу скриншот nmap на моём компьютере (Рисунок 1).

\image{1.png}{nmap}{1}
\FloatBarrier

\subsubsection{Описание и заявленные функциональные возможности}
Разберём работу nmap. Чтобы запустить простое сканирование, достаточно написать nmap и адрес цели. На рисунке 2 показан результат сканирования моего компьютера. Видно, что программа нашла открытый порт 631 и предположила, что на нём запущен сервис ipp. С помощью другой утилиты (netstat) могу выяснить, что это сервис печать cups.
\image{2.png}{nmap localhost}{1}
\FloatBarrier

nmap позволяет задавать подсети и диапазон узлов для сканирования. 
Перечислим полезные режимы:
\begin{itemize}
    \item –sV — определить версию сервисов
    \item –sP — ping сканирование, полезно, чтобы определить наличие узлов в сети
    \item -PN — эта опция будет сканировать даже те хосты, которые блокируют пинги (например, windows-системы с фаерволом)
    \item -A — определение версии ОС
    \item sN/sF/sX — TCP Null, FIN и Xmas сканирования (режимы для определения состояния портов — открыт, закрыт, фильтруется, не фильтруется, смешанный режим)
    \item -p — диапазон портов
\end{itemize}

\subsubsection{Дата выхода и номер последней версии}
Последняя версия nmap (7.94) вышла 20 мая 2023 года. В числе изменений переход на Python 3 в gui-версии, улучшенная работа с MAC и много других улучшений.

\subsubsection{Разработчик}
Утилиту разработал американский программист Gordon Lyon. Впервые программа была опубликована в сентябре 1997 как статья в Phrack Magazine вместе с исходным кодом.

\clearpage

\subsection{Лицензия}
nmap оригинально распространялся под GNU Public License, которая позволяет конечным пользователям запускать, изучать, делиться и модифицировать ПО. Начиная с версии 7.90, nmap распространяется под своей лицензией NPSL.

\subsection{Официальные сайты разработчика и программы}
Скачать и прочитать мануал можно на сайте программы \url{https://nmap.org/}. Создатель программы также поддерживает сайт \url{https://insecure.org/fyodor/}

\subsection{Поддерживаемые операционные системы}
Программа выпускается для всех популярных операционных систем, для различных дистрибутивов Linux, а также доступен исходный код.

\subsection{Установка}
Для установки на Linux:
\begin{itemize}
    \item Arch: pacman -S nmap
    \item Ubuntu: apt install nmap
    \item RPM-based: yay install nmap
\end{itemize}
На Windows нужно скачать установщик с сайта \url{https://nmap.org/download.html}. Аналогично происходит установка на Mac OS.

\subsubsection{Системные требовани}
Особенных требований нет. Windows поддерживается с 7 версии. На Linux для многих типов сканирования нужен root-доступ.

\subsubsection{Особенности установки}
На Windows установка сопровождается установкой драйверов npcap.

\subsection{Первоначальная настройка программы}
Не требуется.

\subsection{Демонстрация функциональных возможностей}
Просканируем специальный хост, предоставляемый разработчиками nmap (Рисунки 3-5). Простое сканирование показывает четыре открытых порта и сервисы, которые обычно на них запущены. OS Type сканирование показывает более подробную информацию, например, что порт 80 фильтруется (то есть защищён фаерволом), а также пытается угадать операционную систему (Linux). Service info сканирование показывает версии OpenSSH сервера.

\image{31.png}{simple}{1}
\FloatBarrier

\image{32.png}{OS type}{1}
\FloatBarrier

\image{33.png}{Service info}{1}
\FloatBarrier

Просканируем два хоста ВШЭ (Рисунки 6-7). Видим, что на хосте hse.ru открыты два порта, которые требуются для протоколов HTTP/HTTPS. Операционная система, предположительно, BSD. На хосте lms.hse.ru найден ещё один закрытый порт 113. Операционная система — Linux.
\image{34.png}{hse.ru}{1}
\FloatBarrier

\image{35.png}{lms.hse.ru}{1}
\FloatBarrier


\subsubsection{Входные и выходные данные}
Входные данные получает из стандартного ввода или из текстового файла, в котором хосты разделены пробелом, табуляцией или новой строкой. Вывод направляется в стандартный вывод.

\subsection{Ссылки на интернет-ресурсы, посвященные программе}
nmap посвящено множество публикаций в интернете. Можно начать с того, что официальный мануал очень удобный для прочтения (\url{https://nmap.org/book/man.html}, на русском языке: \url{https://nmap.org/man/ru/})
\clearpage

\subsection{Вывод}

\subsubsection{Решает ли программа заявленные задачи}
Программа является стандартом для сканирования сети и применяется многими специалистами в силу своей гибкости и надёжности, а также простоты использования.

\subsubsection{Наличие, отсутствие критических проблем при использовани}
За всё время использования nmap не сталкивался с проблемами.

\subsubsection{Удобство интерфейса}
Стандартный интерфейс представляет собой командную строку, что является наиболее гибким  и стандартным подходом в Linux. Интерфейс в Windows (Zenmap) также хорошо работает и предоставляет некоторые дополнительные полезные функции, в частности, визуализацию топологии сети.

\subsubsection{Впечатления от использования программы}
Доволен.


\clearpage

\section{Выводы о проделанной работе}
Я приобрёл навыки работы с приложениями, применяющимися в сфере обеспечения информационной безопасности, на примере утилиты nmap.

\clearpage