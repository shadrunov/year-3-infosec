\section{Цель работы}
Цель: изучить программный комплекс «Стахановец» и приобрести навыки работы с комплексом, настройки политик и мониторинга активности. 

\section{Ход работы}
\subsection{Установка комплекса}
Для разворачивания комплекса понадобится виртуальная машина. Попробуем развернуть с помощью опции Быстрая установка «в один клик» (Windows). В таком режиме база и серверная часть устанавливаются автоматически, что подходит для небольшой конфигурации и знакомства с комплексом.

Установка серверной части осуществляется автоматически (Рисунки 1-2).
\image{11.png}{Установка серверного компонента}{1}
\image{12.png}{Завершение установки}{1}
\FloatBarrier

Затем в ручном режиме устанавливаем клиентский компонент. После установки видим сообщение о запуске подслушивания (Рисунки 3-4). 

\image{13.png}{Установка клиентского компонента}{1}
\image{14.png}{Уведомление пользователя можно скрыть}{0.6}
\FloatBarrier

После установки администратор может войти в два компонента — босс-онлайн и босс-офлайн (Рисунок 5).
\image{15.png}{Босс}{1}
\FloatBarrier
\clearpage


\subsection{Исследование возможностей для обеспечения информационной безопасности}
Выберем функциональные возможности для защиты информационной безопасности и контроля персонала.

\subsubsection{Запрет использования программ}

Можно указать списки приложений для запрета или разрешения соответственно. Каждое приложение должно указываться с новой строки и представлять собой либо полный путь к исполняемому файлу, либо только сам exe-файл без пути, или описание приложения (название из его оригинального поля Description). В случае попадания приложения под запрет запуска будет выдано сообщение в трее на машине пользователя и само приложение будет закрыто. 

Например, так можно заблокировать запуск потенциально нежелательных программ, например, мессенджеров, в которых работник может переслать файлы за пределы организации, или такое устаревшее программное обеспечение, как Internet Explorer (Рисунок 6).

\image{21.png}{Блокировка iexplorer.exe}{1}
\FloatBarrier

\subsubsection{DLP для печати документов}

Если пользователь будет совершать то или иное действие с определенными объектами, в тексте которых присутствуют одно или несколько совпадений из списка чувствительности, то будет сформировано событие, которое может быть далее записано в отчет "События" и выдано мгновенное уведомление в БОСС-Онлайн. События настраиваются на вкладке "События". Также существует возможность запретить те или иные действия.

Для настройки DLP для печати документов включим соответствующую галочку в глобальных настойках. Система будет срабатывать на выражения из списка чувствительности, например, "Совершенно секретно".

Для работы DLP также необходимо включить соответствующие опции мониторинга на одноименных вкладках настроек (то есть Мониторинг -> Печать на принтере), а также включить теневое копирование на клиенте и на сервере.

\image{31.png}{DLP для печати документов}{1}
\FloatBarrier


\subsubsection{DLP для документов в буфере обмена}
DLP для документов в буфере обмена работает схожим образом с DLP для печати. Настраиваем, как на рисунках 8-9. Не забываем включить мониторинг.

Этот способ защиты может быть полезен для защиты конфиденциальности данных, точнее, для защиты от нежелательного копирования.

\image{41.png}{DLP для текста в буфере обмена}{1}
\image{42.png}{Мониторинг текста в буфере обмена}{1}
\FloatBarrier


\subsubsection{Ограничения в критичных программах}
Если в компании используются приложения или сайты, копирование или фотографирование данных из которых крайне нежелательно, то имеет смысл использовать ограничения в критичных программах. При запуске пользователем программы/сайта из списка будут происходить запреты/действия из отмеченных. Также при возникновении запрета будет сгенерировано событие. 

Выбираем запретить PrintScreen и запретить буфер обмена в программе Word (Рисунок 10).

\image{51.png}{Ограничения в критичных программах}{1}
\FloatBarrier

\subsubsection{Нетипичное поведение}
На этой Нетипичное поведение настраивается возможность отслеживания нетипичного поведения сотрудника по ряду критериев. Можно задать интервал отслеживания — время наблюдения, в течение которого ведется подсчет всех остальных критериев. Если в течение данного времени любой из критериев превысил указанное в настройках значение, то происходит событие. Также необходимо включение соответствующих настроек на вкладках "Теневое копирование", "Файловые операции", "Буфер обмена", "Отправка файлов", "Программы/сайты".

Подобный поведенческий анализ позволяет выявлять подозрительную активность сотрудников и предотвращать действия, направленные на нарушение конфиденциальности или целостности каких-либо защищаемых данных.

Установим контроль за копированием и удалением любых файлов (Рисунок 11).
\image{61.png}{Нетипичное поведение}{1}
\FloatBarrier


\subsubsection{Пользовательское время}
Перейдём к настройкам контроля сотрудников. Для включения опции контроля пользовательского времени нужно зайти на соответствующую вкладку в глобальных настройках и включить эту функцию (Рисунок 12). Эта опция позволяет следить, в какое время был активен пользователь и в каких программах работал, а затем строить отчёты.

\image{71.png}{Пользовательское время}{1}
\FloatBarrier

\subsubsection{Программы/сайты}
Вкладка программы/сайты позволяет подключить детализацию по приложениям и сайтам (Рисунок 13).

\image{81.png}{Пользовательское время}{1}
\FloatBarrier

\subsubsection{Снимки с экранов}
Вкладка Снимки с экранов настраивает параметры выгрузки скриншотов с рабочих станций (Рисунок 14).

\image{91.png}{Снимки с экранов}{1}
\FloatBarrier

Наиболее важные опции на этой вкладке: получать и передавать на сервер снимки с экранов — нужно включить, чтобы получать скриншоты в системе босс. Делать снимки каждые 20 минут — частота снимков. Также можно изменить качество скришнотов, чтобы отрегулировать нагрузку на сеть и занимаемое место.



\clearpage


\subsection{Тестирование возможностей}

\subsubsection{Запрет использования программ}
Протестируем работу запрета. Попробуем запустить iexplorer.exe. Программа открывается, затем принудительно закрывается, а в трее появляется сообщение (Рисунок 15).
\image{22.png}{Блокировка iexplorer.exe}{1}
\FloatBarrier

\subsubsection{DLP для печати документов}
Протестируем работу DLP для печати документов. Для этого создадим текстовый документ, содержащий чувствительную фразу "Совершенно секретно" (Рисунок 16). Отправим документ на печать. В результате срабатывает уведомление в босс-онлайн (Рисунок 17).
\image{32.png}{Секретный документ}{0.7}
\image{33.png}{Уведомление для босса}{1}
\FloatBarrier

\subsubsection{DLP для документов в буфере обмена}
Протестируем работу DLP для документов в буфере обмена. Для этого найдём текстовый документ, содержащий чувствительную фразу "Совершенно секретно" (Рисунок 18). Скопируем документ. В результате срабатывает уведомление в босс-онлайн.
\image{43.png}{Уведомление для босса}{1}
\FloatBarrier

\subsubsection{Ограничения в критичных программах}
Протестируем работу ограничений в критичных программах. Для этого запустим критичный процесс (winword.exe) и попробуем сделать скриншот. Видим, что защита сработала и скриншот сделать не удалось, а в боссе-онлайн появилось уведомление (Рисунок 19).
\image{52.png}{Уведомление для босса}{0.9}
\FloatBarrier

\subsubsection{Нетипичное поведение}
Протестируем работу нетипичного поведения. Удалим несколько файлов с рабочего стола. Видим, как в боссе срабатывает алерт (Рисунок 20).
\image{62.png}{Уведомление для босса}{1}
\FloatBarrier

\subsubsection{Пользовательское время}
Протестируем работу пользовательского времени. Для этого в системе босс-офлайн построим отчёты по времени. Примеры на рисунках ниже.
\image{72.png}{Пользовательское время}{1}
\image{75.png}{Переработал, пока делал лабу}{1}
\FloatBarrier

\subsubsection{Программы/сайты}
Протестируем работу функции Программы/сайты. Для этого в системе босс-офлайн построим отчёты по времени в приложениях. Примеры на рисунках ниже.
\image{73.png}{Программы/сайты}{1}
\image{74.png}{Программы/сайты (упрощённый)}{1}
\FloatBarrier

\subsubsection{Снимки с экранов}
Протестируем работу функции Снимки с экранов. По сути эта функция встроена во многие другие возможности, в том числе на срабатывание событий присылается скриншот. Пример текущего скриншота на рисунке 26.
\image{92.png}{Скриншот}{1}
\FloatBarrier
\clearpage

\section{Выводы о проделанной работе}
Я изучил и приобрёл навыки работы с комплексом Стахановец, настроил мониторинг событий информационной безопасности и контроль за сотрудниками. 

\clearpage