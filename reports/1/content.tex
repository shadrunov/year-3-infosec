\section{Задание на практическую работу}
Ознакомиться с функциями безопасности Windows 7, улучшениями и приложениями. Изучить брандмауэр Windows.



\section{Ход работы}
\subsection{Пользователи}
Создаём двух новых пользователей user1 и user2 (рисунки 1 и 2). 
\image{11.png}{User1}{1}
\image{1.png}{Два пользователя в оснастке}{1}
\FloatBarrier

\subsection{Группы безопасности}
Создаём две группы безопасности group1 и group2 (рисунок 3). Включаем каждого пользователя в свою группу. 
\image{2.png}{Группы безопасности}{1}
\FloatBarrier

\subsection{Разграничение доступа}
Создаём четыре каталога со следующими правами (рисунки 4-7):
\begin{itemize}
    \item каталог1: только чтение для пользователя1, чтение, создание и изменение файлов
    для пользователя2;
    \item каталог2: нет доступа для пользователя1, только чтение для пользователя2;
    \item каталог3: только чтение для группы1, только чтение для группы2;
    \item каталог4: полный доступ для группы1, создание и изменение файлов для группы2;
\end{itemize}
\image{31.png}{каталог1}{0.86}
\image{32.png}{каталог2}{0.86}
\image{33.png}{каталог3}{0.86}
\image{34.png}{каталог4}{0.86}
\FloatBarrier
Видно, что у пользователя user1 нет доступа к dir2 (рисунок 8).
\image{35.png}{У пользователя1 нет доступа к каталог2}{0.86}
\FloatBarrier

\subsection{Шаблон безопасности системы}
Создаём шаблон безопасности системы, указав параметры по заданию: вести журнал паролей, минимальная длина пароля, пароль должен отвечать требованиям сложности, пороговое значение блокировки, аудит входа в систему, аудит доступа к объектам, аудит событий входа в систему, аудит управления учетными записями (рисунки 9-10).

\image{41.png}{Новый шаблон}{0.95}
\image{42.png}{Заданы параметры аудита}{0.95}
\FloatBarrier

\subsection{Анализ параметров безопасности системы}
Проводим анализ параметров безопасности системы по заданному шаблону и выявляем различия между шаблоном и существующими параметрами (рисунки 11-13). Видно, что все заданные параметры отличаются от стандартных в системе.

\image{51.png}{Различия между шаблоном и текущей настройкой политики паролей}{1}
\image{52.png}{Различия между шаблоном и текущей настройкой политики блокировки учётной записи}{0.9}
\image{53.png}{Различия между шаблоном и текущей настройкой политики аудита}{0.95}
\FloatBarrier

\subsection{Настройка параметров безопасности по шаблону}
Производим настройку параметров безопасности по заданному шаблону. Для этого нажимаем на \texttt{Анализ и настройка безопасности} и выбираем \texttt{Настроить компьютер...} (рисунок 14). После этого шаблон автоматически применяется к системе. На рисунке 15 видно, что различия между шаблоном и текущей настройкой исчезли.

\image{61.png}{Настроить компьютер...}{1}
\image{62.png}{Различия между шаблоном и текущей настройкой исчезли}{0.9}
\FloatBarrier

\subsection{Работа с журналом безопасности}
Далее выполняем действия, влияющие на безопасность системы, и отслеживаем логи в соответствующем журнале аудита. 

На рисунках 16-17 видно, что мы задаём пароль, удовлетворяющий требованиям локальной политики, а также запись об успешном доступе к учётной записи в журнале.
\image{71.png}{Задаём безопасный пароль для user1}{0.95}
\image{72.png}{Запись в журнале безопасности операционной системы}{0.95}

На рисунках 18-20 мы задаём пароль, не удовлетворяющий требованиям локальной политики (слишком короткий). Запись в журнале говорит о неудачной попытке изменения учётных данных (рисунок 20).
\image{73.png}{Задаём небезопасный пароль для user2}{0.9}
\image{74.png}{Ошибка}{0.9}
\image{75.png}{Запись в журнале безопасности операционной системы}{0.9}
\FloatBarrier

На рисунках 21-22 производим вход в систему под учетной записью пользователя user1 с ошибочными данными (неправильным паролем). Находим связанное с этим событие в журнале безопасности операционной системы.
\image{76.png}{Указываем неверный пароль}{0.55}
\image{77.png}{Запись в журнале безопасности операционной системы}{0.9}
\FloatBarrier

На рисунках 23-24 производим вход в систему под учетной записью пользователя user1 с верными данными. Находим связанное с этим событие в журнале безопасности операционной системы.
\image{78.png}{Успешный вход под учетной записью пользователя user1}{0.7}
\image{79.png}{Запись в журнале безопасности операционной системы}{0.8}
\FloatBarrier

\subsection{Брандмауэр Windows}
Настраиваем правила в брандмауэре Windows по заданию (рисунки 25-13):
\begin{itemize}
    \item запрет исходящего трафика для подключения к сайтам (html-страницам);
    \item запрет входящего трафика для удалённого подключения к рабочей станции по протоколу RDP (утилита \texttt{mstsc})
\end{itemize}
\image{83.png}{Запрет html-страниц (порты 80 и 443)}{0.86}
\image{84.png}{Запрет RDP}{0.86}
\FloatBarrier

Перед этим проверяем работу браузера и RDP (всё работает, рисунки 27-28).
\image{82.png}{Работа браузера (порт 443)}{0.65}
\image{81.png}{Работа RDP-подключения к машине win7}{0.89}
\FloatBarrier

После включения правил подключения нет (рисунки 29-30).
\image{86.png}{Запрет html-страниц (порты 80 и 443)}{0.75}
\image{85.png}{Запрет RDP}{0.8}
\FloatBarrier


\subsection{UAC}
Настраиваем параметры управления учетными записями пользователей на уведомление при любом изменении параметров компьютера (рисунок 31). После применения самой строгой настройки попытка изменить параметры компьютера вызывает уведомление и требует административный доступ к системе (рисунок 32).
\image{91.png}{Включение высокого уровня оповещений}{1}
\image{92.png}{Уведомление о внесении изменений в систему}{0.9}
\FloatBarrier

\subsection{Архивация и восстановление}
С помощью инструмента «Архивация и восстановление» выполняем архивацию
(резервное копирование) каталога с файлами \texttt{lab} (рисунок 33). Результат отображается в свойствах каталога (рисунок 34).
\image{93.png}{Выбор папки для архивации}{0.8}
\image{94.png}{Результат архивации}{0.8}
\FloatBarrier


\subsection{Защитник Windows}
В качестве дополнительного средства защиты рассмотрен защитник Windows. Для начала требуеся его включить (рисунок 35). Затем можно провести антивирусную проверку (рисунок 36).
\image{101.png}{Включение защитника}{0.6}
\image{102.png}{Быстрая проверка}{0.7}
\FloatBarrier


\clearpage



\section{Выводы о проделанной работе}
В рамках данной работы я ознакомился с функциями безопасности Windows 7, улучшениями и приложениями, изучил брандмауэр Windows.