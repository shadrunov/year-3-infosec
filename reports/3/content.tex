\section{Цель работы}
Целью данной лабораторной работы является изучение программно-аппаратных средств стеганографии. Изученить возможности и практику работы с ImageSpyer и RedJPEG. Написать собственное программное средство для внедрения сообщения.


\section{Ход работы}

\subsection{Image Spyer G2}
Установим Image Spyer G2 на виртуальную машину. Окно программы приведено на рисунке 1.

\image{11.png}{Окно программы}{1}
\FloatBarrier

Сначала встроим текстовый файл в картинку (рисунки 2-7). Затем встроим полученную картинку в оригинал (рисунок 8).

\image{12.png}{Выбор начальной картинки}{0.75}
\image{13.png}{Начальная картинка}{0.75}
\image{14.png}{Выбор текстового файла}{0.75}
\image{15.png}{Задание пароля}{0.75}
\image{16.png}{Успешное встраивание}{0.75}
\image{17.png}{Сохранение файла}{0.75}
\image{18.png}{Другой файл встроили в картинку}{0.75}
\FloatBarrier

Результат встраивания отображён на рисунках 9 и 10.

\image{41.png}{Текстовый файл в картинке}{0.7}
\image{42.png}{Картинка в картинках}{0.7}
\FloatBarrier


\subsection{RedJPEG}
Установим RedJPEG на виртуальную машину. Окно программы приведено на рисунке 11.

\image{21.png}{Окно программы RedJPEG}{1}
\FloatBarrier

Встроим текстовый файл в картинку (рисунки 12-13). 

\image{22.png}{Текстовый файл}{0.83}
\image{23.png}{Успешно встроено}{0.83}

\FloatBarrier

Результат встраивания отображён на рисунке 14.

\image{43.jpg}{Текстовый файл в картинке}{0.6}
\FloatBarrier


\subsection{Программа на python}
Сделаем программу для встраивания текста вручную. 

\textbf{Алгоритм встраивания}:
\begin{itemize}
    \item открыть файл. прочитать заголовок, вычислить сдвиг (offset).
    \item считать строку для встраивания. привести к байтовому представлению.
    \item снова открыть файл на чтение, открыть файл для записи, прочитать и записать начало файла до сдвига.
    \item каждый бит из байтового представления длины строки вставить в LSB картинки, записать.
    \item для каждого байта строки каждый бит записать в LSB очередного байта картинки, записать.
    \item записать остаток картинки без изменения.
\end{itemize}

\textbf{Алгоритм извлечения}:
\begin{itemize}
    \item открыть файл. прочитать заголовок, вычислить сдвиг (offset).
    \item считать строку для встраивания. привести к байтовому представлению.
    \item снова открыть файл на чтение, открыть файл для записи, прочитать и записать начало файла до сдвига.
    \item каждый бит из байтового представления длины строки вставить в LSB картинки, записать.
    \item для каждого байта строки каждый бит записать в LSB очередного байта картинки, записать.
    \item записать остаток картинки без изменения.
\end{itemize}

Код программ приведён в приложении. Результат работы программы и сравнение картинки в hex-редакторе приведены на рисунках 15-16.

\image{31.png}{Работа программы}{1}
\image{32.png}{Сравнение в hex-редакторе}{1}
\FloatBarrier

\FloatBarrier
\clearpage

\section{Выводы о проделанной работе}
Я изучил программно-аппаратных средства стеганографии: ImageSpyer и RedJPEG, а также написал собственное программное средство для внедрения сообщения.

\clearpage