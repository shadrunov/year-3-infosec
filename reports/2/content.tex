\section{Цель работы}
Целью данной лабораторной работы является обучение студентов базовым навыками обращения с ОС класса Linux и основам обеспечения безопасности данных систем.


\section{Ход работы}

\subsection{Подготовка виртуальной машины}

Создадим виртуальную машину в гипервизоре \texttt{gnome-boxes} (\texttt{kvm}). Процесс изображён на рисунках 1-3.

\image{1.png}{Установка ВМ}{0.8}
\image{2.png}{Установка ВМ}{0.8}
\image{3.png}{Установка ВМ}{0.82}
\FloatBarrier


\subsection{Настройка сетевых интерфейсов}

Для настройки сетевых интерфейсов используется утилита \texttt{netplan}. По умолчанию настроено получение адреса по \texttt{DHCP} (рисунок 4).

\image{4.png}{\texttt{DHCP}}{0.7}
\FloatBarrier

Настроим статический адрес. Для этого в файл \texttt{/etc/netplan/} \linebreak \texttt{00-installer-config.yaml} пропишем следующую конфигурацию (рисунок 5). Применим конфигурацию и проверим результат (рисунок 6).


\image{5.png}{Static}{0.7}
\image{6.png}{Применение конфигурации}{0.7}
\FloatBarrier

Вернёмся к автоматической конфигурации.


\subsection{\texttt{adduser}}

Для создания пользователей в Linux существует утилита \texttt{adduser}. Параметры команды позволяют задать \texttt{uid/gid}, домашний каталог, имя группы, оболочку. Создадим двух пользователей (рисунки 7-8).

\image{21.png}{\texttt{student}}{0.8}
\image{22.png}{\texttt{auditor}}{0.8}
\FloatBarrier

\subsection{\texttt{passwd}}

Для смены пароля учётной записи используется утилита \texttt{passwd}. Для смены своего пароля достаточно выполнить команду \texttt{passwd}, для смены пароля чужой учётной записи нужно выполнить \texttt{sudo passwd <user>} (рисунок 9).

\image{31.png}{Смена пароля}{0.8}
\FloatBarrier

Также утилита позволяет отключить пароль или заставить пользователя сменить пароль при следующем входе или заблокировать аккаунт.


\subsection{mc}

В Linux есть файловый менеджер с псевдографическим интерфейсом \texttt{Midnight Commander}. Для установки используется команда \texttt{sudo apt install mc}.
Рабочее окно \texttt{mc} представлено на рисунке 10.

\image{41.png}{\texttt{mc}}{0.92}
\FloatBarrier

Для поиска файлов нужно перейти в меню \texttt{Command > Find file} (рисунок 11). Далее можно ввести имя файла и файл будет найден (рисунки 12-13).

\image{42.png}{\texttt{Command > Find file}}{1}
\image{43.png}{\texttt{File name}}{1}
\image{44.png}{\texttt{Results}}{1}
\FloatBarrier

Далее создадим файл с помощью команды \texttt{touch} (рисунок 14). Новый файл появляется в списке (рисунок 15).

\image{45.png}{Команда \texttt{touch}}{0.8}
\image{46.png}{Новый файл}{0.8}
\FloatBarrier

Для редактирования нужно выбрать кнопку F4 (\texttt{Edit}). Открывается текстовый редактор (рисунок 16).

\image{47.png}{Редактирование}{0.8}
\FloatBarrier

Далее для просмотра разрешений перейдём в меню \texttt{File > Chmod} (рисунок 17). Появится окно с просмотром разрешений (рисунок 18).

\image{48.png}{\texttt{File > Chmod}}{1}
\image{49.png}{\texttt{Chmod}}{1}
\FloatBarrier
\clearpage


\subsection{\texttt{history}}

Для просмотра истории команд можно использовать утилиту \texttt{history}. Она отображает содержание файла \texttt{.bash\_history} (для оболочки \texttt{bash}). В истории команд можно искать с помощью команды \texttt{grep} (рисунки 19-20). Можно настроить различные параметры, например, глубину хранения, с помощью переменных окружения \texttt{HISTSIZE} (сколько команд хранится в текущей истории) и \texttt{HISTFILESIZE} (сколько команд из текущей истории записывается в файл \texttt{.bash\_history}). 

\image{51.png}{Просмотр истории}{1}
\image{52.png}{Ограничение истории до 2 команд}{1}
\FloatBarrier


\subsection{\texttt{grep, cat, ls}}

Для манипуляций с файлами есть команды \texttt{grep, cat, ls}. Первая позволяет выбрать строки из файлов по определённому параметру. Можно использовать вместе с оператором \texttt{pipe} (рисунки 21-22). Также можно использовать регулярные выражения.

\image{61.png}{Команды grep}{0.5}
\image{62.png}{Регулярные выражения}{0.5}
\FloatBarrier

Команда cat читает последовательно файлы и выдает их содержимое в стандартный выходной поток. Одна из полезных опций — \texttt{-n} — позволяет выдавать порядковый номер строки перед каждой строкой (рисунок 23).

\image{63.png}{Команда \texttt{cat}}{0.8}
\FloatBarrier

Команда \texttt{ls} выводит список файлов. Полезные опции: \texttt{-l} — вывод списком, \texttt{-a} — вывод скрытых файлов (рисунок 24).

\image{64.png}{Команда \texttt{ls}}{0.85}
\FloatBarrier


\subsection{\texttt{chmod}}

Для манипуляций с разрешениями используется команда \texttt{chmod}. Для выставления разрешений используются буквенный или цифровой синтаксисы (рисунок 25).

\image{71.png}{Команда \texttt{chmod}}{0.75}
\FloatBarrier


\subsection{\texttt{arp}}

Команда \texttt{arp} работает с arp-таблицей, то есть соответствием \texttt{ip-} и \texttt{mac-}адресов. Без ключей выводит кэш таблицы (рисунок 25). Также можно добавлять или удалять строки (ключи \texttt{-d} и \texttt{-s}) или выбирать форматирование (рисунок 26).

\image{81.png}{Команда arp}{1}
\image{82.png}{Команда arp с ключами}{1}
\FloatBarrier

\clearpage

\subsection{ip}

Команда \texttt{ip} позволяет настроить сеть и сетевые интерфейсы. У неё очень много параметров и подкоманд. \texttt{ip link} используется для отображения и изменения сетевых интерфейсов (рисунок 27). \texttt{ip monitor} позволяет просматривать сетевые события (рисунок 28). \texttt{ip route} позволяет работать с таблицей маршрутизации (рисунок 29). Кроме того, можно управлять адресами ({ip addr}) и arp-таблицей ({ip neigh}).

\image{91.png}{Команда \texttt{ip link}}{1}
\image{92.png}{Команда \texttt{ip monitor}}{1}
\image{93.png}{Команда \texttt{ip route}}{0.8}
\FloatBarrier


\subsection{\texttt{ping}}

Команда \texttt{ping} позволяет проверять доступность хостов. Для работы нужно выполнить команду \texttt{ping} и адрес хоста (рисунок 31). Аргументы команды отображены на рисунке 32.

\image{101.png}{Команда \texttt{ping}}{1}
\image{102.png}{Команда \texttt{ping}}{1}
\FloatBarrier


\subsection{\texttt{traceroute}}

Команда \texttt{traceroute} позволяет отобразить маршрут до удалённых хостов. Механизм работы основан на поле ip-пакета \texttt{ttl}. Примеры работы команды приведены на рисунке 33.

\image{111.png}{Команда \texttt{traceroute}}{1}
\FloatBarrier


\subsection{\texttt{netstat}}

Команда \texttt{netstat} позволяет отобразить сетевую статистику, используемые порты, интерфейсы и процессы, их использующие. Примеры работы команды приведены на рисунках 34-35.

\image{121.png}{Команда \texttt{netstat}}{1}
\image{122.png}{Команда \texttt{netstat}}{0.7}
\FloatBarrier


\subsection{\texttt{nslookup}}

Команда \texttt{nslookup} позволяет работать с dns-записями, запрашивать ip-адрес по имени хоста и наоборот. Примеры работы команды приведены на рисунке 36.

\image{131.png}{Команда \texttt{nslookup}}{0.55}
\FloatBarrier


\subsection{Работа с \texttt{python}}

Для работы с \texttt{python} требуется установить интерпретатор языка и менеджер пакетов \texttt{pip} (\texttt{sudo apt install python3}). После этого можно запустить скрипт на языке \texttt{python} (рисунок 37). Работа с менеджером пакетов показана на рисунке 38.

\image{141.png}{Команда \texttt{python}}{0.8}
\image{142.png}{Команда \texttt{pip}}{0.9}
\FloatBarrier


\subsection{Механизм безопасности}

Один из механизмов безопасности, предусмотренный в системах \texttt{Linux}, это удалённый доступ без пароля по \texttt{ssh}. Позволяет отключить доступ к аккаунту по паролю и оставить доступ только по ключу. Это предпочтительнее, так как пароль может быть подобран злоумышленником, либо забыт владельцем, ключ имеет заданную сложность и лишён этих недостатков. На рисунке 39 показано, как передать ключ на машину, а также как получить удалённую консоль по \texttt{ssh}.

\image{151.png}{Команда \texttt{ssh}}{0.9}
\FloatBarrier


\clearpage

\section{Выводы о проделанной работе}
В рамках данной работы я освоил базовые навыки работы с \texttt{Linux} и основам обеспечения безопасности этих систем.

\clearpage